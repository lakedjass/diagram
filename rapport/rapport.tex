\documentclass[11pt,a4paper]{report}
\usepackage[utf8]{inputenc}
\usepackage[french]{babel}
\usepackage[T1]{fontenc}
\usepackage{amsmath}
\usepackage{amsfonts}
\usepackage{amssymb}
\begin{document}
\begin{titlepage}
\newcommand{\HRule}{\rule{\linewidth}{0.5mm}}
\center
\textsc{\LARGE
Ecole Nationale Superieure Polytechnique de Maroua
} \\[1cm]
\textsc{\Large
Departement d'Informatique et Telecommunications
} \\[1cm]
\textsc{\large
\underline{UE} : Cryptographie avancee \\ \underline{CODE} : SEC519
} \\[1cm]
\HRule \\[0.4cm]
{ \huge \bfseries Rapport des Travaux Diriges 1  \\[0.15cm]}
\HRule \\[1.5cm]
KOUAM CHEKAM LEOPOLD JUNIOR \\
18A0265P \\[1cm]
MOGO KAMDEM ROOSEVELT\\
18A0308P \\[1cm]
\today \\ [1cm]
\end{titlepage}
\newpage
\tableofcontents
\chapter*{Introduction}
\chapter{Étude de l'existant}
\chapter{Cahier de charge}
\section{Contexte et définition du problème}
 Dans le cadre de l'unité d'enseignement Spécialisation en cryptographie 2, nous nous sommes intéressé à la manière donc les personnes et entreprises gèrent leurs mots de passe mais surtout à la robustesses de ceux-ci.
 \section{Objectifs du projet}
 Dans le but de faciliter la gestion des mots de passes aux utilisateurs/entreprises, nous avons opté concevoir et produire une solution de gestion de mot de passe robuste.
 \section{Spécifications des besoins}
 \subsection{Besoins fonctionnels}
 Il est question pour nous dans cette section de définir les besoins fonctionnels de notre gestionnaire de mot de passe.un besoin fonctionnel est un action ou un comportement permis dans le système. Notre système permettra :
 \subsubsection{Cas d'utilisation personnel}
 \begin{itemize}
 \item Authentifier un utilisateur
 \item Ajouter/Editer/Supprimer un mot de passe
 \item generer un mot de passe
 \item generer u mot de passe robuste
 \item tester la robutesse d'un mot de passe
 \end{itemize}
 
 \subsubsection{Cas d'utilisation en entreprise}
 TODO
 
 \subsection{Besoins non fonctionnels}
 un besoin non fonctionnel est un exigence propre au système niveau performance, matériel ... mais surtout les contraintes d'implémentation. Ainsi le système doit :
 \begin{itemize}
 \item simple a utiliser
 \item un délai de réponse très court
 \item fonctionner suivant l'architecture client-serveur (trois niveaux) 
 \item évolutif
 \end{itemize}
 \subsection{Méthode adoptée}
 \subsection{Budgétisation}
\chapter{Implémentation}
\chapter*{Conclusion et perspectives}
\end{document}